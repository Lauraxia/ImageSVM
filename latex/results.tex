\section {Testing \& Results of Support Vector Classification}

\chapter{Classification of Testing Data}
To determine optimal performance, we wanted to test the system setup with different numbers of training points.

\chapter{Overall Effect}

Over the numerous types of parameters we could specify for the support vector machine to operate on, the order of effectiveness can be determined by the relative accuracy increase through its inclusion into the training data.

\begin{enumerate}
  \item L*ab (Direct Pixel Value)
  \item L*ab (Radial Blur Pixel Value)
  \item XY (Pixel Location)
\end{enumerate}

The pixel location had a surprisingly opposite reaction to the effectiveness of the system upon adding its values to the training data.

\chapter{Outcome}

By limiting the data sent to the training set for the support vector machine, it was possible to observe 90\% and above performance for even small amounts of training data. The position of these training points had a greater effect on the results at when the total training points was lower. As expected, the results are very sensitive to the position of the training points -- as more of the points will become support vectors that are further away from the optimal hyperplane. The sensitivity of the accuracy decreased drastically with increased number of point.


